segregation

r2
pseudo-r2

table of access vs income

cross diagram <- for individual areas / areas of interest

https://www.scimagojr.com/journalsearch.php?q=130140&tip=sid&clean=0

placed based planning

individual income - income support - unemp benefits
shift to place based support

mgwr - 

group areas -

housing stress 30/%

sydney global arc

rban policy formed a prominent part of its election platform. The new Prime Minister said, in a political speech: "... no amount of wealth redistribution through higher wages or lower taxes can really offset the inequalities imposed by the physical nature of cities. Increasingly, a citizen's real standard of living, the health of himself and his family, his children's opportunities for education and self-improvement, his access to employment opportunities, his ability to enjoy the nation's resources for recreation and culture ... are determined not by his income, but by where he lives" (Whitlam, quoted by Logan et al, 1975, pages 106-107). 'Inequality' had been redefined partly as a function of residential location—where people live and how
accessible they are to the jobs, goods, and services of the city (Catley and McFarlane,
1974; Sandercock, 1975). 


\begin{equation}
    w_{ij} =\left (1 - \frac{d_{ij}^2}{b^{2}}  \right )^{2},\; if \; d_{ij} < r, \; 0 \; otherwise
\end{equation}

\begin{equation}
y_{i} = \beta_{bw0} + \sum_{k} \alpha_{bwk} x_{ik} + \varepsilon_{i}
\end{equation}
MGWR

